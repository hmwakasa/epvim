%! LuaLaTeX 文書
\documentclass[unicode,aspectratio=1610,colorlinks,handout]{beamer}
\usetheme[numbering=fraction,block=fill]{metropolis}
\usefonttheme{professionalfonts}

\hypersetup{linkcolor=,urlcolor=teal}

\usepackage{luatexja}
\ltjsetparameter{jacharrange={-2,-3,-8}}
\usepackage[no-math,match,deluxe]{luatexja-preset}

\usepackage{textcomp}
\usepackage{luatexja-otf}

\usepackage{graphicx,xcolor}
\usepackage{pxrubrica}
\usepackage{tcolorbox}
\usepackage{bxwareki}
\usepackage{indentfirst}

\usepackage{listings}
\lstset{
	doubleletterspace,
	basicstyle={\small\ttfamily},
	backgroundcolor={\color[gray]{.9}},
	commentstyle={\color[gray]{.5}},
	numbers=left,
	numberstyle=\tiny,
}

\usepackage{tikz}
\usetikzlibrary{plotmarks}

\usepackage{scsnowman} %☃
\usepackage{bxokumacro}
\usepackage{menukeys}
\renewmenumacro{\keys}[+]{shadowedroundedkeys}
\usepackage{bxrawstr} % https://zrbabbler.hatenablog.com/entry/20181222/1545495849
\usepackage[twemoji-pdf]{bxcoloremoji} % https://github.com/zr-tex8r/BXcoloremoji

\usepackage{mflogo}

\usepackage{bxtexlogo}
\bxtexlogoimport{*,**}

\usepackage[T1]{fontenc}
\usepackage{mathtools,amssymb,mathrsfs,mathabx}
\usepackage[math]{iwona}
\usepackage[euler-digits]{eulervm}
\allowdisplaybreaks[4]

%%%%%%%%%% 商用フォントを用いているので、自分でタイプセットする場合は適当に以下を変えてください
\setmainfont[
	Ligatures=TeX,
	BoldFont=FOT-RodinNTLGPro-EB,
	ItalicFont=FOT-RodinNTLGPro-EB,
]{FOT-RodinNTLGPro-B}
\setsansfont[
	Ligatures=TeX,
	BoldFont=FOT-RodinNTLGPro-EB,
	ItalicFont=FOT-RodinNTLGPro-EB,
]{FOT-RodinNTLGPro-B}
\setmainjfont[
	Ligatures=TeX,
	CharacterWidth=Proportional,
	JFM=prop,
	BoldFont=FOT-RodinNTLGPro-EB,
	ItalicFont=FOT-RodinNTLGPro-EB,
]{FOT-RodinNTLGPro-B}
\setsansjfont[
	Ligatures=TeX,
	CharacterWidth=Proportional,
	JFM=prop,
	BoldFont=FOT-RodinNTLGPro-EB,
	ItalicFont=FOT-RodinNTLGPro-EB,
]{FOT-RodinNTLGPro-B}
\setmonofont[
	Ligatures=TeXReset,
]{HackGen}
\setmonojfont[
	Ligatures=TeXReset,
]{HackGen}

\newfontfamily\errorfont[
	Ligatures=TeX,
]{TsukuQMinLStd-L}

\newfontfamily\nishikifonta[
	Ligatures=TeX,
]{Nishiki-teki}
\newjfontfamily\nishikifontj[
	Ligatures=TeX,
]{Nishiki-teki}
\newcommand{\nishikifont}{\nishikifonta\nishikifontj}

%%%%%%%%%% 以下は自前コマンド

\newcommand{\centeralign}[1]{\rule{0pt}{0pt}\hfill#1\hfill\rule{0pt}{0pt}}
\newcommand{\typecommand}[1]{\colorbox{darkgray}{{\ttfamily\color{lime}#1}}}
\newtcolorbox{hmcommandbox}{
		sharp corners,left=0mm,right=0mm,top=0mm,bottom=0mm,
		colframe=darkgray,
		colback=darkgray,
		fontupper=\ttfamily,
		colupper=lime,
}

\setlength{\parskip}{2ex}

\title{便利に使う Vim 入門}
\author{北海道大学理学院 宇宙理学専攻 M1 人見祥磨}
\date{\warekitoday}

\begin{document}
\rubyusejghost % \jruby で和文ゴースト

\frame{\maketitle}

\begin{frame}
	\section{Vim とは何か}
\end{frame}

\begin{frame}
	\frametitle{Vim とは}
	\begin{itemize}
		\item \textbf{V}i \textbf{im}proved
		\item テキストエディタ vi を拡張したもの
		\item 現在では、vi で vim(の機能を制限したもの)が起動することが多い
			\begin{hmcommandbox}
				\% vi --version\\
				VIM - Vi IMproved 8.1 {\scriptsize 以下略}
			\end{hmcommandbox}
	\end{itemize}
	
	\begin{block}{テキストエディタとは}
		テキストファイルを編集するためのソフトウェア
		\begin{description}
			\item[スクリーンエディタ] テキストファイルの中身を画面に表示しながら、
				カーソルを移動させてテキストを編集するもの
			\item[ラインエディタ] テキストファイルの中身を表示\emph{せず}、行単位
				でコマンドを利用しながら編集するもの
		\end{description}
		
		vi は ラインエディタ ex の流れを汲む
	\end{block}
\end{frame}

\begin{frame}
	\frametitle{Vim の使い方}
	\begin{block}{Vim を使いこなすのに参考になるもの}
		\begin{description}
			\item[\typecommand{vimtutor}] Vim を一切知らない人が、一通りの機能を扱える
				様になるための公式チュートリアル
			\item[ヘルプファイル] 正確で詳しい情報が載っている
		\end{description}
	\end{block}
	
	\begin{block}{ヘルプファイルの表示方法}
		Vim ノーマルモードで、\typecommand{:help <調べたいもの>} と打ち込むことで表示できる
		
		日本語のヘルプファイルを導入するにはひと手間必要
		
		\url{https://vim-jp.org/vimdoc-ja/index.html} からも日本語ドキュメントを参照できる
	\end{block}
\end{frame}

\begin{frame}
	\frametitle{Vim のモード}
	Vim は複数のモードを持つ
	
	複数のモードを行き来してテキストの編集を行う
	
	\begin{center}
		\begin{tikzpicture}[x=.6\textwidth,y=.2\textwidth,>=stealth]
			\node(N)at(0,0){ノーマルモード};
			\node(I)at(1,0){挿入モード};
			\node(V)at(0,-1){ヴィジュアルモード};
			\node(C)at(1,-1){コマンドラインモード};
			\draw[->](N)--(I)node[midway,above]{i, a, I, A, O, o など};
			\draw[->](N)--(V)node[midway,left]{v, V など};
			\draw[->](N)--(C)node[midway,above right]{:};
		\end{tikzpicture}
	\end{center}
	
	Esc キーでノーマルモードに戻る
\end{frame}

\begin{frame}
	\frametitle{それぞれのモードの役割}
	\begin{tabular}{ll}
		ノーマルモード&カーソル移動、ヤンク、貼り付け、削除などをするモード\\
		挿入モード&文字入力などをするモード\\
		ビジュアルモード&範囲を選択するモード\\
		コマンドラインモード&コマンドを実行するモード
	\end{tabular}
	
	\begin{block}{Ex モード}
		Ex モードというモードもある(ノーマルモードで Q で移行)
		
		コマンドを複数続けて実行するモード
		
		<Esc> ではノーマルモードに戻れない
		
		:visual でノーマルモードに戻る
	\end{block}
\end{frame}

\begin{frame}
	\section{Vim の設定}
\end{frame}

\begin{frame}
	\frametitle{Vim の設定ファイル}
	Vim の設定はコマンドラインモードで行うことができる
	
	例: \typecommand{:syntax on} でカラーリングをオンにできる
	
	よく使う設定を毎回設定するのは面倒
	
	\textasciitilde/.vimrc ファイルにコマンドを書き込んでおくと、Vim 起動時に読み込んでくれる
	
	\begin{block}{defaults.vim}
		vimrc ファイルが存在しない場合に読み込まれるもの
		
		新規の Vim ユーザーにとって使いやすい設定がまとめられている
		
		コマンドラインモードでのタブ補完や、マウスの利用設定などがある
		
		defaults.vim の詳細は \typecommand{:help defaults.vim-explained}
	\end{block}
\end{frame}

\begin{frame}[fragile]
	\frametitle{.vimrc の例}
	\begin{lstlisting}
" から始まる行はコメント
" defaults.vim を確実に読み込ませる
unlet! skip_defaults_vim

" defaults.vim を読み込む
source $VIMRUNTIME/defaults.vim

" 日本語のヘルプファイルが利用できるときは利用する
set helplang=ja,en

"Tab を明示的に表示する。(\ のあとにはスペース)
set list
set listchars=tab:^\ 
	\end{lstlisting}
\end{frame}

\begin{frame}
	\section{ノーマルモードでの操作}
\end{frame}

\begin{frame}
	\frametitle{モード間の移動}
	\begin{center}
		\begin{tabular}{cll}
			コマンド&意味&移行するモード\\
			\hline
			a&カーソルの後ろに文字を追加 (append)&挿入モード\\
			A&行の後ろに文字を追加&挿入モード\\
			i&カーソルの前に文字を挿入 (insert)&挿入モード\\
			I&行の先頭に文字を挿入&挿入モード\\
			o&カーソルの次の行に新しい行を追加 (open?)&挿入モード\\
			O&カーソル位置に新しい行を追加&挿入モード\\
			s&カーソル位置の文字を削除して挿入モード&挿入モード\\
			S&行を削除して挿入モード&挿入モード\\
			v&文字単位のヴィジュアルモード&ヴィジュアルモード\\
			V&行単位のヴィジュアルモード&ヴィジュアルモード\\
			C-v&矩形ヴィジュアルモード&ヴィジュアルモード\\
			r&カーソル位置の文字を 1 文字置き換え (replace)&置換モード\\
			R&カーソル下の文字をどんどん置き換えるモード&置換モード\\
		\end{tabular}
	\end{center}
\end{frame}

\begin{frame}
	\frametitle{モーション}
	カーソルを移動させるコマンド
	
	範囲を表すこともできる(後述)
	
	数字と組み合わせて繰り返し回数を指定できる\\
	i.e. 2w → 2 単語移動
	
	\begin{center}
		\begin{tabular}{cl}
			コマンド&意味\\
			\hline
			h j k l&← ↓ ↑ →\\
			w e b&単語単位の移動 (word, end, back)\\
			\$ \textasciicircum&行末・行頭へ移動\\
			G gg&ファイル先頭・末尾へ移動\\
			\%&対応するカッコへ移動
		\end{tabular}
	\end{center}
	
	その他の移動コマンドは \typecommand{:help quickref}
\end{frame}

\begin{frame}
	\frametitle{オペレーター}
	\begin{center}
		\begin{tabular}{cl}
			コマンド&意味\\
			\hline
			y&ヤンク (yank) (コピー)\\
			d&削除 (delete)\\
			c&削除して挿入モードに移行 (change)
		\end{tabular}
	\end{center}
	
	オペレーターは範囲を表すコマンドと組み合わせる\\
	i.e. c\$ → 行末まで削除して挿入モードに移行
	
	\begin{block}{テキストオブジェクト}
		a(ひとつの ``a'' まとまり)または i (inner) と組み合わせて範囲を指定する
		\begin{center}
			\begin{tabular}{cl}
				コマンド&意味\\
				\hline
				iw&カーソルがある単語\\
				a( a)&( ) を含めた全体
			\end{tabular}
		\end{center}
		
		ci\{ とすれば \{ \} の中身を削除して挿入モードに移行
	\end{block}
\end{frame}

\begin{frame}
	\frametitle{その他のコマンド}
	\begin{center}
		\begin{tabular}{cl}
			コマンド&意味\\
			\hline
			D&行末まで削除 (d\$)\\
			C&行末まで削除して挿入モードに移行 (c\$)\\
			dd&1 行削除\\
			yy&1 行ヤンク\\
			p&カーソルの後ろに貼り付け\\
			P&カーソルの手前に貼り付け\\
			x&1 文字削除\\
			J&行を連結\\
			u&アンドゥ\\
			C-r&リドゥ\\
			.&繰り返し\\
			ZZ&変更を保存して終了(変更がなければ何もしない)(:x)\\
			ZQ&何もせず終了 (:q!)
		\end{tabular}
	\end{center}
\end{frame}

\begin{frame}
	\section{ヴィジュアルモード}
\end{frame}

\begin{frame}
	\frametitle{ヴィジュアルモードとは}
	範囲を選択するモード
	
	範囲を選択したあとに c d y などを組み合わせる
	
	矩形ヴィジュアルモード (C-v) では、複数行への挿入 (I) や編集もできる
	
	> や < でインデントを増減することができる
\end{frame}

\begin{frame}
	\section{コマンドラインモード}
\end{frame}

\begin{frame}[fragile]
	\frametitle{コマンドラインモードでのコマンド}
	\begin{center}
		\begin{tabular}{cl}
			コマンド&意味\\
			\hline
			:e filename&filename を編集\\
			:w&上書き保存\\
			:help&ヘルプを表示\\
			:sp :vsp&画面を分割表示\\
			:tabnew&タブページを利用\\
			:!cmd&外部コマンド cmd を使用\\
			:r!cmd&cmd の結果を挿入
		\end{tabular}
	\end{center}
	
	.vimrc で自前のコマンドを定義できる
	
	\begin{lstlisting}
" :pLaTeX で、編集しているファイルを platex で処理する
" % は編集しているファイル名に展開される :r で拡張子を取り除く
command pLaTeX !ptex2pdf %:r
	\end{lstlisting}
\end{frame}
\end{document}
